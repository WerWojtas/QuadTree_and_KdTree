\documentclass{lab}

\usepackage[utf8]{inputenc}
\usepackage{graphicx}
\usepackage{tocloft}
\usepackage{amssymb}
\usepackage{xcolor}
\usepackage{enumitem}
\usepackage{amsmath} 
\usepackage{animate}
\usepackage{graphicx}
\usepackage[T1]{fontenc} 
\usepackage[top=0.5in, bottom=0.5in, left=0.5in, right=0.5in]{geometry}
\usepackage{mathrsfs}
\usepackage{hyperref}
\usepackage{wrapfig}
\usepackage{indentfirst}
\usepackage{mathtools}
\usepackage[export]{adjustbox}
\usepackage{scrextend}
\usepackage{animate}
\usepackage{float}
\newcommand\tab[1][1cm]{\hspace*{#1}}
\geometry{
    margin=0.5in,
    left=2.2cm,
    right=2.2cm,
    headheight=17pt,
    headsep=1cm, % Increase the margin from the top for the header
    includehead,
    includefoot
}

\usepackage[T1]{fontenc}
\usepackage[polish]{babel}
\usepackage[utf8]{inputenc}
\usepackage{amsmath}
\usepackage{listings}
\usepackage{hyperref}
\usepackage{pythonhighlight}
\usepackage{amssymb}
\usepackage{mathtools}
\usepackage{array}
\usepackage[margin=0.5in,
    left=2.2cm,
    right=2.2cm,
    headheight=17pt,
    includehead, includefoot]{geometry}
    
%% nagłówek
\usepackage{fancyhdr}
\usepackage{tikz}
\usepackage{pgfplots}
\usepackage{pgfplotstable}
\pagestyle{fancy}
\fancyhf{}
\rhead{Algorytmy Geometryczne}
\lhead{Wyszukiwanie geometryczne}
\rfoot{\thepage}

%% Podmiana \part na obecne wyświetlanie
\makeatletter
\renewcommand\@part[2][]{%
  \ifnum \c@secnumdepth >-2\relax
    \refstepcounter{part}%
    \addcontentsline{toc}{part}{\thepart#2}%
  \else
    \addcontentsline{toc}{part}{#2}%
  \fi
  \markboth{}{}%
  {\raggedright % Align title to the left
   \interlinepenalty \@M
   \normalfont
   \ifnum \c@secnumdepth >-2\relax
     \Large\bfseries \thepart\hspace{1em}%
   \fi
   \huge \bfseries #2\par}%
  \@endpart
}
\makeatother

\renewcommand\thepart{\Roman{part}. }

\addtokomafont{labelinglabel}{\sffamily}

\renewcommand{\cftsecleader}{\cftdotfill{\cftdotsep}}
\renewcommand{\cftsubsecleader}{\cftdotfill{\cftdotsep}}


\begin{document}

\begin{figure*}
    \centering
    \includegraphics{agh.png}
\end{figure*}
\title{\Huge \textbf{Wyszukiwanie Geometryczne}\\QuadTree i KdTree \\ \huge Dokumentacja}
\author{Weronika Wojtas, Radosław Rolka\\Algorytmy geometryczne\\Informatyka WI AGH, II rok}
\date{}

\maketitle
\newpage
\tableofcontents
\thispagestyle{fancy} 
\newpage

\part{Wstęp}
\section{Wymagania techniczne}
...
\section{Środowisko wykonawcze}
...
\section{Zawarość programu}
...

\part{Dokumentacja}
\section{Visualizer}
...
\section{QuadTree}
...
\section{KdTree}
...
\section{Tests}
...

\part{Poradnik}
\section{Testy}
...
\section{Wizualizacja}
...

\part{Sprawozdanie}
\section{Wstęp teoretyczny}
...
\subsection{QuadTree}
...
\subsection{KdTree}
...
\section{Porównanie wyników dla różnych danych wejściowych}
...
\section{Porównanie wyników dla różnych wartości capacity}
...chodzi o quadtree i max ilość punktów w węźle
\section{Testowanie dla różnej ilości wymiarów}
...chodzi o kdtree (k-dimension tree)
\section{Podsumowanie}
...

\end{document}

