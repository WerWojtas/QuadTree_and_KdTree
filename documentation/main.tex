\documentclass{lab}

\usepackage[utf8]{inputenc}
\usepackage{graphicx}
\usepackage{tocloft}
\usepackage{amssymb}
\usepackage{xcolor}
\usepackage{enumitem}
\usepackage{amsmath} 
\usepackage{animate}
\usepackage{graphicx}
\usepackage[T1]{fontenc} 
\usepackage[top=0.5in, bottom=0.5in, left=0.5in, right=0.5in]{geometry}
\usepackage{mathrsfs}
\usepackage{hyperref}
\usepackage{wrapfig}
\usepackage{indentfirst}
\usepackage{mathtools}
\usepackage[export]{adjustbox}
\usepackage{scrextend}
\usepackage{animate}
\usepackage{float}
\newcommand\tab[1][1cm]{\hspace*{#1}}
\geometry{
    margin=0.5in,
    left=2.2cm,
    right=2.2cm,
    headheight=17pt,
    headsep=1cm, % Increase the margin from the top for the header
    includehead,
    includefoot
}

\usepackage[T1]{fontenc}
\usepackage[polish]{babel}
\usepackage[utf8]{inputenc}
\usepackage{amsmath}
\usepackage{listings}
\usepackage{hyperref}
\usepackage{pythonhighlight}
\usepackage{amssymb}
\usepackage{mathtools}
\usepackage{array}
\usepackage{xpatch}

\xpretocmd{\part}{\setcounter{section}{0}}{}{}
\usepackage[margin=0.5in,
    left=2.2cm,
    right=2.2cm,
    headheight=17pt,
    includehead, includefoot]{geometry}
    
%% nagłówek
\usepackage{fancyhdr}
\usepackage{tikz}
\usepackage{pgfplots}
\usepackage{pgfplotstable}
\pagestyle{fancy}
\fancyhf{}
\rhead{Algorytmy Geometryczne}
\lhead{Wyszukiwanie geometryczne}
\rfoot{\thepage}

%% Podmiana \part na obecne wyświetlanie
\makeatletter
\renewcommand\@part[2][]{%
  \ifnum \c@secnumdepth >-2\relax
    \refstepcounter{part}%
    \addcontentsline{toc}{part}{\thepart#2}%
  \else
    \addcontentsline{toc}{part}{#2}%
  \fi
  \markboth{}{}%
  {\raggedright % Align title to the left
   \interlinepenalty \@M
   \normalfont
   \ifnum \c@secnumdepth >-2\relax
     \Large\bfseries \thepart\hspace{1em}%
   \fi
   \huge \bfseries #2\par}%
  \@endpart
}
\makeatother

\renewcommand\thepart{\Roman{part}. }

\addtokomafont{labelinglabel}{\sffamily}

\renewcommand{\cftsecleader}{\cftdotfill{\cftdotsep}}
\renewcommand{\cftsubsecleader}{\cftdotfill{\cftdotsep}}


\begin{document}

\begin{figure*}
    \centering
    \includegraphics{resources/agh.png}
\end{figure*}
\title{\Huge \textbf{Wyszukiwanie Geometryczne}\\QuadTree i KdTree \\ \huge Dokumentacja}
\author{Weronika Wojtas, Radosław Rolka\\Algorytmy geometryczne\\Informatyka WI AGH, II rok}
\date{2023}

\maketitle
\newpage
\tableofcontents
\thispagestyle{fancy} 
\newpage

\part{Wstęp}
\section{Informacje wstępne}
Celem tej dokumentacji jest przedstawienie zaawansowanych technik wyszukiwania geometrycznego, z naciskiem na implementację i zastosowanie KdTree oraz QuadTree oraz ich porównanie.

Projekt został zaimplementowany w języku Python 3.9 z wykorzystaniem aplikacji Jupyter Notebook. Kod źródłowy wraz z dokumentacją zamieszczony jest w repozytorium GitHub.

\section{Wymagania techniczne}
\begin{itemize}
    \item python >= 3.9
    \item numpy >= 1.25.2
    \item pandas >= 2.0.3
    \item matplotlib >= 3.7.2
    \item notebook >= 6.5.4
    \item functools (Python Standard Library)
    \item math (Python Standard Library)
    \item copy (Python Standard Library)
\end{itemize}

\section{Środowisko wykonawcze}
\begin{table}[h]
\centering
\begin{tabular}{|c|c|c|}
\hline
 & Komputer 1 & Komputer 2 \\
\hline
Procesor & Intel(R) Core(TM) I5-10300H 2.50GHz & TODO \\
\hline
System operacyjny & Microsoft Windows 10 64bit ver 22H2 & TODO \\
\hline
\end{tabular}
\caption{Dane techniczne maszyn wykonawczych}
\end{table}
Porównanie efektywności struktur zostało wykonane na Komputerze 1.
\section{Zawarość programu}
Program składa się z następujących plików:
\begin{itemize}
    \item KdTree.py
    \item QuadTree.py
    \item visualizer/
    \item utilities/
    \item tests/
    \item documentation/
    \item comparator/
    TODO: napisać jak to wygląda w praktyce
\end{itemize}

\part{Dokumentacja}
\section{Utilities}
Ten moduł zapewnia podstawowe klasy elementów geometrycznych: punkt (Point) i prostokąt (Rectangle), które unifikują dane wejściowe do obu implementacji drzew oraz zapewniają walidację wprowadzonych danych.
\subsection{Point}
Reprezentacja niemutowalnego punktu o dodatniej liczbie wymiarów.
\begin{itemize}
    \item \textbf{Point(point)}: 
    
    Konstruktor punktu przyjmuje iterowalny obiekt jako argument i zapisuje jego wartości jako kolejne wymiary dla punktu.
    
    Wywołuje ValueError("Point must have at least one dimension") jeśli obiekt jest pusty.
    \item \textbf{\_\_eq\_\_(self, other)}:
    
    Porównuje punkty na podstawie ich wymiarów, zwraca True jeśli wymiary są równe, w przeciwnym wypadku False. Umożliwia porównywanie punktów oraz innych iterowalnch struktur za pomocą operatorów == i !=.
    \item \textbf{\_\_hash\_\_(self)}:
    
    Zwraca hash punktu, umożliwiając jego użycie jako klucza w słownikach.

    \item \textbf{\_\_str\_\_(self) / \_\_repr\_\_(self)}:
    
    Zwraca string reprezentujący punkt w postaci (x1, x2, ..., xn).

    \item \textbf{\_\_len\_\_(self)}:
    
    Zwraca liczbę wymiarów punktu.

    \item \textbf{\_\_getitem\_\_(self, key)}:
    
    Zwraca wymiar o indeksie key, umożliwiając dostęp do wymiarów punktu za pomocą operatora nawiasów kwadratowych.

    \item \textbf{point(self)}:
    
    @property Zwraca listę zawierającą kopię wymiarów punktu.

    \item \textbf{x(self)}:
    
    @property Zwraca pierwszy wymiar punktu.

    \item \textbf{y(self)}:
    
    @property Zwraca drugi wymiar punktu, jeśli istnieje.

    \item \textbf{follows(self, other)}:
    
    Zwraca True jeśli punkt jest większy od other w każdym wymiarze, w przeciwnym wypadku False. Zezwala na porównywanie obiektów iterowalnych.
    Wywołuje ValueError("Can only compare Points of the same dimensionality") jeśli wymiary punktów nie są równe.

    \item \textbf{precedes(self, other)}:
    
    Zwraca True jeśli punkt jest mniejszy od other w każdym wymiarze, w przeciwnym wypadku False. Zezwala na porównywanie obiektów iterowalnych.

    Wywołuje ValueError("Can only compare Points of the same dimensionality") jeśli wymiary punktów nie są równe.

    \item \textbf{distance(self, other)}:
    
    Zwraca odległość między punktami. Zezwala na porównywanie obiektów iterowalnych.

    Wywołuje ValueError("Can only compare Points of the same dimensionality") jeśli wymiary punktów nie są równe.

    \item \textbf{minimum(self, other)}:

    Zwraca punkt zawierający najmniejsze wartości z obu punktów. Zezwala na porównywanie obiektów iterowalnych.

    Wywołuje ValueError("Can only compare Points of the same dimensionality") jeśli wymiary punktów nie są równe.

    \item \textbf{maximum(self, other)}:
    
    Zwraca punkt zawierający największe wartości z obu punktów. Zezwala na porównywanie obiektów iterowalnych.

    Wywołuje ValueError("Can only compare Points of the same dimensionality") jeśli wymiary punktów nie są równe.    
\end{itemize}

\subsection{Rectangle}
Reprezentacja niemutowalnego prostokąta o dodatniej liczbie wymiarów, zapisanego jako dwa przeciwległe wierzchołki: lewy dolny (lowerleft) i prawy górny (upperright).
\begin{itemize}
  \item \textbf{Rectangle(lowerleft, upperright)}:
  
  Konstruktor prostokąta przyjmuje dwa punkty (lub obiekty iterowalne) jako argumenty i zapisuje je jako przeciwległe wierzchołki prostokąta.

  Wywołuje ValueError("Points must have the same dimensionality") jeśli wymiary punktów nie są równe.

  Wywołuje ValueError("LowerLeft point must precede the UpperRight point") jeśli któryś z wymiarów punktu lowerleft jest większy niż odpowiadający mu wymiar punktu upperright.

  \item \textbf{\_\_eq\_\_(self, other)}:
  
  Porównuje prostokąty na podstawie ich wierzchołków, zwraca True jeśli wierzchołki są równe, w przeciwnym wypadku False. Umożliwia porównywanie za pomocą operatorów == i !=.

  \item \textbf{\_\_hash\_\_(self)}:
  
  Zwraca hash prostokąta, umożliwiając jego użycie jako klucza w słownikach.

  \item \textbf{\_\_str\_\_(self) / \_\_repr\_\_(self)}:
  
  Zwraca string reprezentujący prostokąt w postaci [(x1, x2, ..., xn), (y1, y2, ..., yn)].

  \item \textbf{\_\_len\_\_(self)}:
  
  Zwraca liczbę wymiarów prostokąta.

  \item \textbf{lowerleft(self)}:
  
  @property Zwraca punkt lowerleft.

  \item \textbf{upperright(self)}:
  
  @property Zwraca punkt upperright.

  \item \textbf{from\_points(cls, points)}:
  
  @classmethod Tworzy najmniejszy prostokąt zawierający wszystkie punkty (lub obiekty iterowalne) z listy points.

  Wywołuje ValueError("Cannot create a Rectangle from an empty list of points") jeśli lista points jest pusta.

  Wywołuje ValueError("Points must have the same dimensionality") jeśli wymiary punktów nie są równe.

  \item \textbf{does\_intersect(self, other)}:
  
  Sprawdza czy dwa prostokąty przecinają się (mają conajmniej jeden punkt wspólny), zwraca True jeśli tak, w przeciwnym wypadku False.

  Wywołuje ValueError("Can only check intersection with another Rectangle") jeśli other nie jest prostokątem.

  Wywołuje ValueError("Can only check intersection with a Rectangle of the same dimensionality") jeśli wymiary prostokątów nie są równe.

  \item \textbf{contains(self, object)}:
  
  Sprawdza czy punkt lub prostokąt jest całkowicie zawarty w prostokącie, zwraca True jeśli tak, w przeciwnym wypadku False. Jeśli obiekt nie jest prostokątem, funkcja uznaje go za punkt.

  \item \textbf{divide(self, dimension, value)}:
  
  Dzieli prostokąt na dwa prostokąty wzdłuż wymiaru dimension i wartości value, zwraca krotkę dwóch prostokątów.

  Wywołuje ValueError("Dimension must be between 0 and {len(self)-1}") jeśli wymiar dimension jest mniejszy niż 0 lub większy niż liczba wymiarów prostokąta - 1.

  Wywołuje ValueError("Value must be between {self.lowerleft[dimension]} and {self.upperright[dimension]}") jeśli wartość value jest mniejsza niż wartość wymiaru dimension punktu lowerleft lub większa niż wartość wymiaru dimension punktu upperright.

  \item \textbf{intersection(self, other)}:
  
  Zwraca prostokąt będący częścią wspólną dwóch prostokątów lub None jeśli prostokąty się nie przecinają.

  Wywołuje ValueError("Can only compute intersection with another Rectangle") jeśli other nie jest prostokątem.

  Wywołuje ValueError("Can only compute intersection with a Rectangle of the same dimensionality") jeśli wymiary prostokątów nie są równe.
\end{itemize}
\section{QuadTree}
...
\section{KdTree}
...
\section{Comparator}
...
\section{Tests}
...
\section{Visualizer}
...

\part{Instrukcja}
\section{Testy}
...
\section{Wizualizacja}
...

\part{Sprawozdanie}
\section{Wstęp teoretyczny}
...
\subsection{QuadTree}
...
\subsection{KdTree}
...
\section{Porównanie wyników dla różnych danych wejściowych}
...
\section{Porównanie wyników dla różnych wartości capacity}
...chodzi o quadtree i max ilość punktów w węźle
\section{Testowanie dla różnej ilości wymiarów}
...chodzi o kdtree (k-dimension tree)
\section{Podsumowanie}
...

\end{document}